\documentclass[12pt]{article}
%% arXiv paper template by Flip Tanedo


%%%%%%%%%%%%%%%%%%%%%%%%%%%%%
%%%  THE USUAL PACKAGES  %%%%
%%%%%%%%%%%%%%%%%%%%%%%%%%%%%

\usepackage{amsmath}         % \
\usepackage{amssymb}         %  | AMS Packages for math
\usepackage{amsfonts}        % /
\usepackage{graphicx}        % Graphics
 

%%%%%%%%%%%%%%%%%%%%%%%%%%%%%%%%%
%%%  UNUSUAL PACKAGES        %%%%
%%%  Uncomment as necessary. %%%%
%%%%%%%%%%%%%%%%%%%%%%%%%%%%%%%%%

\usepackage{lipsum}        % block of text (formatting test)
%\usepackage{color}         % \color{...}, colored text
%\usepackage{slashed}       % \slashed{k}
%\usepackage{framed}        % boxed remarks
%\usepackage{subcaption}    % subfigures; subfig depreciated
%\usepackage{mathrsfs}      % Weinberg-esque letters
%\usepackage{paralist}      % compactitem
%\usepackage{multirow}      % multiple row elements in a table
%\usepackage{cite}          % grouping citations (incompatible with collref)
%\usepackage{booktabs}      % tables
%\usepackage{nicefrac}      % fractions in tables,
%\usepackage{youngtab}	    % Young Tableaux
%\usepackage{arydshln} 	    % dashed lines in arrays and tables
%\usepackage{appendix}      % subappendices
%\usepackage{pifont}        % check marks
%\usepackage{bbm}           % \mathbbm{1} incompatible with XeLaTeX 


%%%%%%%%%%%%%%%%%%%%%%%%%%%%%%%%%%%%%%%%%%%
%%%  FLIP'S CUSTOM PACKAGES            %%%%
%%%  These are in separate .sty files  %%%%
%%%%%%%%%%%%%%%%%%%%%%%%%%%%%%%%%%%%%%%%%%%

\usepackage{flip-acronyms} % HEP acronyms in small caps, e.g. \GeV
\usepackage{tikzfeynman}   % Flip's Feynman Diagrams


%%%%%%%%%%%%%%%%%%%%%%%%%%%%%%%%%%%%%%%%%%%%%%%
%%%  PAGE FORMATTING and (RE)NEW COMMANDS  %%%%
%%%%%%%%%%%%%%%%%%%%%%%%%%%%%%%%%%%%%%%%%%%%%%%


\usepackage[margin=2cm]{geometry}   % reasonable margins
\graphicspath{{figures/}}	        % set directory for figures
\numberwithin{equation}{section}    % set equation numbering
\renewcommand{\tilde}{\widetilde}   % tilde over characters
\renewcommand{\vec}[1]{\mathbf{#1}} % vectors are boldface


\newcommand{\dbar}{d\mkern-6mu\mathchar'26}    % for d/2pi
\newcommand{\ket}[1]{\left|#1\right\rangle}    % <#1|
\newcommand{\bra}[1]{\left\langle#1\right|}    % |#1>
\newcommand{\Xmark}{\text{\sffamily X}}        % cross out


% Commands for temporary comments
\newcommand{\comment}[2]{\textcolor{red}{[\textbf{#1} #2]}}
\newcommand{\flip}[1]{{\tt \color{red} [Flip: {#1}]}}
\newcommand{\email}[1]{\href{mailto:#1}{#1}}


\newenvironment{institutions}[1][2em]{\begin{list}{}{\setlength\leftmargin{#1}\setlength\rightmargin{#1}}\item[]}{\end{list}}


%%%%%%%%%%%%%%%%%%%%%%%%%%%%%%%%%%%%%%%%%%%%%%
%%%  TIKZ COMMANDS FOR EXTERNAL DIAGRAMS  %%%%
%%%  requires -shell-escape               %%%%
%%%%%%%%%%%%%%%%%%%%%%%%%%%%%%%%%%%%%%%%%%%%%%

%\usetikzlibrary{external}
%\tikzexternalize[prefix=tikz/] % folder for external pdfs



%%%%%%%%%%%%%%%%%%%
%%%  HYPERREF  %%%%
%%%%%%%%%%%%%%%%%%%

% This package has to be at the end; can lead to conflicts
\usepackage[
	colorlinks=true,
	citecolor=black,
	linkcolor=black,
	urlcolor=blue,
	hypertexnames=false]{hyperref}



%%%%%%%%%%%%%%%%%%%%%
%%%  TITLE DATA  %%%%
%%%%%%%%%%%%%%%%%%%%%


\begin{document}

\thispagestyle{empty}
\begin{center}

    {\huge \textbf{Sample Feynman Diagrams in TikZ} \\
    \large \textsc{Vol.~III: Blob Diagrams} }

    \vskip .7cm

    { \bf Flip Tanedo } 
    \\ \vspace{-.2em}
    { \tt
    \footnotesize
    \email{flip.tanedo@uci.edu} 
    }
	
    \vspace{-.2cm}

    \begin{institutions}[2.25cm]
    \footnotesize
	\vspace*{0.05cm}
	{\it 
	    Department of Physics \& Astronomy, 
	    University of California, 
	    Irvine, \textsc{ca} 92697
	    }   
    \end{institutions}

\end{center}



%%%%%%%%%%%%%%%%%%%%%
%%%  ABSTRACT    %%%%
%%%%%%%%%%%%%%%%%%%%%

\begin{abstract}
\noindent This is collection of useful sample Feynman diagrams and pieces typeset in TikZ. See Volume I for background information.
\end{abstract}




%%%%%%%%%%%%%%%%%%%%%
%%%  THE MEAT    %%%%
%%%%%%%%%%%%%%%%%%%%%

% Use \input if you have separate files.
% \include is `smarter' (creates separate aux files for each tex file) 
% and hence more efficient, but it automatically puts a page break
% between included files. Input doesn't do this.

\section{Direct Detection Blob}

	\begin{tikzpicture}[line width=1.5 pt, scale=2]
		\draw[fermion](145:1) -- (145:.3cm);
			\node at (145:1.15) {$\chi$};
		\draw[fermion](215:1) -- (215:.3cm);
			\node at (215:1.2) {SM};
		\draw[fermionbar](35:1) -- (35:.3cm);
			\node at (35:1.15) {$\chi$};
		\draw[fermionbar](-35:1) -- (-35:.3cm);
			\node at (-35:1.2) {SM};
		\draw[fill=black] (0,0) circle (.3cm);
		\draw[fill=white] (0,0) circle (.29cm);
		\begin{scope}
	    	\clip (0,0) circle (.3cm);
	    	\foreach \x in {-.9,-.8,...,.3}
				\draw[line width=1 pt] (\x,-.3) -- (\x+.6,.3);
	  	\end{scope}
	 \end{tikzpicture}	


\section{'t Hooft operator}


	\begin{tikzpicture}[line width=1.5 pt, scale=2.5]
		\draw[fermionbar] (0:.75) -- (25:.29);
		\draw[fermionnoarrow] (0:.75) -- (-25:.29);
		\draw[scalar] (0:.75) -- (0:1.25);
		\begin{scope}[shift={(1.25,0)}]
			\draw (125:.1) -- (-55:.1);
			\draw (55:.1) -- (-125:.1);			
		\end{scope}
		\draw[fermionbar] (60:.75) -- (85:.29);
		\draw[fermionnoarrow] (60:.75) -- (35:.29);
		\draw[scalar] (60:.75) -- (60:1.25);
		\begin{scope}[shift={(60:1.25)}, rotate=60]
			\draw (125:.1) -- (-55:.1);
			\draw (55:.1) -- (-125:.1);			
		\end{scope}
		\draw[fermionbar] (120:.75) -- (145:.29);
		\draw[fermionnoarrow] (120:.75) -- (95:.29);
		\draw[scalar] (120:.75) -- (120:1.25);
		\begin{scope}[shift={(120:1.25)}, rotate=120]
			\draw (125:.1) -- (-55:.1);
			\draw (55:.1) -- (-125:.1);			
		\end{scope}
		\draw[fermionbar] (300:.75) -- (325:.29);
		\draw[fermionnoarrow] (300:.75) -- (275:.29);
		\draw[scalar] (300:.75) -- (300:1.25);
		\begin{scope}[shift={(300:1.25)}, rotate=300]
			\draw (125:.1) -- (-55:.1);
			\draw (55:.1) -- (-125:.1);			
		\end{scope}
		\draw[fill=black] (320:1) circle (.01);
		\draw[fill=black] (325:1) circle (.01);
		\draw[fill=black] (330:1) circle (.01);
		\draw[fill=black] (335:1) circle (.01);
		\draw[fill=black] (340:1) circle (.01);
		\draw[fill=black] (80:1) circle (.01);
		\draw[fill=black] (85:1) circle (.01);
		\draw[fill=black] (90:1) circle (.01);
		\draw[fill=black] (95:1) circle (.01);
		\draw[fill=black] (100:1) circle (.01);
		\draw[fill=black] (0,0) circle (.3cm);
		\draw[fill=white] (0,0) circle (.29cm);
		\begin{scope}
	    	\clip (0,0) circle (.3cm);
	    	\foreach \x in {-.9,-.8,...,.3}
				\draw[line width=1 pt] (\x,-.3) -- (\x+.6,.3);
	  	\end{scope}
		% \node at (0,0) {I};
		\node at (278:.6) {$\lambda$};
		\node at (323:.6) {$Q$};
		\node at (98:.6) {$\lambda$};
		\node at (143:.6) {$\overline Q$};
		\node at (290:1) {$\tilde Q$};
		\node at (130:1) {$\tilde Q$};
		\node at (300:1.5) {$v$};
		\node at (120:1.5) {$v$};
		\node at (60:1.5) {$v$};
		\node at (0:1.5) {$v$};
		\draw[fermionnoarrow] (180:.75) -- (180:.29);
		\draw[fermionnoarrow] (240:.75) -- (240:.29);
		\draw[fermion] (160:1) -- (180:.75);
		\draw[scalar] (180:.75) -- (200:1);
		\begin{scope}[shift={(200:1)}, rotate=60]
			\draw (125:.1) -- (-55:.1);
			\draw (55:.1) -- (-125:.1);			
		\end{scope}
		\draw[fermion] (260:1) -- (240:.75);
		\draw[scalar] (240:.75) -- (220:1);
		\begin{scope}[shift={(220:1)}]
			\draw (125:.1) -- (-55:.1);
			\draw (55:.1) -- (-125:.1);			
		\end{scope}
		\node at (200:1.25) {$v$};
		\node at (220:1.25) {$v$};
		\node at (265:1.2) {$Q$};
		\node at (155:1.2) {$\overline Q$};
	 \end{tikzpicture}
	
	\vspace{2em}
	
	
	
	\begin{tikzpicture}[line width=1.5 pt, scale=2.5]
		\draw[fermionbar] (0:.75) -- (25:.29);
		\draw[fermionnoarrow] (0:.75) -- (-25:.29);
		\draw[scalar] (0:.75) -- (0:1.25);
		\begin{scope}[shift={(1.25,0)}]
			\draw (125:.07) -- (-55:.07);
			\draw (55:.07) -- (-125:.07);			
		\end{scope}
		\draw[fermionbar] (60:.75) -- (85:.29);
		\draw[fermionnoarrow] (60:.75) -- (35:.29);
		\draw[scalar] (60:.75) -- (60:1.25);
		\begin{scope}[shift={(60:1.25)}, rotate=60]
			\draw (125:.07) -- (-55:.07);
			\draw (55:.07) -- (-125:.07);
		\end{scope}
		\draw[fermionbar] (120:.75) -- (145:.29);
		\draw[fermionnoarrow] (120:.75) -- (95:.29);
		\draw[scalar] (120:.75) -- (120:1.25);
		\begin{scope}[shift={(120:1.25)}, rotate=120]
			\draw (125:.07) -- (-55:.07);
			\draw (55:.07) -- (-125:.07);
		\end{scope}
		\draw[fermionbar] (300:.75) -- (325:.29);
		\draw[fermionnoarrow] (300:.75) -- (275:.29);
		\draw[scalar] (300:.75) -- (300:1.25);
		\begin{scope}[shift={(300:1.25)}, rotate=300]
			\draw (125:.07) -- (-55:.07);
			\draw (55:.07) -- (-125:.07);
		\end{scope}
		\draw[fill=black] (320:1) circle (.01);
		\draw[fill=black] (325:1) circle (.01);
		\draw[fill=black] (330:1) circle (.01);
		\draw[fill=black] (335:1) circle (.01);
		\draw[fill=black] (340:1) circle (.01);
		\draw[fill=black] (80:1) circle (.01);
		\draw[fill=black] (85:1) circle (.01);
		\draw[fill=black] (90:1) circle (.01);
		\draw[fill=black] (95:1) circle (.01);
		\draw[fill=black] (100:1) circle (.01);
		\draw[fill=black] (0,0) circle (.3cm);
		\draw[fill=white] (0,0) circle (.29cm);
		\begin{scope}
	    	\clip (0,0) circle (.3cm);
	    	\foreach \x in {-.9,-.8,...,.3}
				\draw[line width=1 pt] (\x,-.3) -- (\x+.6,.3);
	  	\end{scope}
		% \node at (0,0) {I};
		% \node at (278:.6) {$\lambda$};
		% \node at (323:.6) {$Q$};
		% \node at (98:.6) {$\lambda$};
		% \node at (143:.6) {$\overline Q$};
		% \node at (290:1) {$\tilde Q$};
		% \node at (130:1) {$\tilde Q$};
		% \node at (300:1.5) {$v$};
		% \node at (120:1.5) {$v$};
		% \node at (60:1.5) {$v$};
		% \node at (0:1.5) {$v$};
		\draw[fermionnoarrow] (180:.75) -- (180:.29);
		\draw[fermionnoarrow] (240:.75) -- (240:.29);
		\draw[fermion] (160:1) -- (180:.75);
		\draw[scalar] (180:.75) -- (200:1);
		\begin{scope}[shift={(200:1)}, rotate=60]
			\draw (125:.07) -- (-55:.07);
			\draw (55:.07) -- (-125:.07);
		\end{scope}
		\draw[fermion] (260:1) -- (240:.75);
		\draw[scalar] (240:.75) -- (220:1);
		\begin{scope}[shift={(220:1)}]
			\draw (125:.07) -- (-55:.07);
			\draw (55:.07) -- (-125:.07);
		\end{scope}
		% \node at (200:1.25) {$v$};
		% \node at (220:1.25) {$v$};
		% \node at (265:1.2) {$Q$};
		% \node at (155:1.2) {$\overline Q$};
	 \end{tikzpicture}
	
	

\section{Hadronic interactions}

\begin{tikzpicture}[line width=1.5 pt, scale=2]
	% \draw[fermion](145:1) -- (145:.3cm);
	% 	\node at (145:1.15) {$\chi$};
	% \draw[fermion](215:1) -- (215:.3cm);
	% 	\node at (215:1.2) {SM};
	% \draw[fermionbar](35:1) -- (35:.3cm);
	% 	\node at (35:1.15) {$\chi$};
	% \draw[fermionbar](-35:1) -- (-35:.3cm);
	% 	\node at (-35:1.2) {SM};
	% \draw[fill=black] (0,0) circle (.3cm);
	% \draw[fill=white] (0,0) circle (.29cm);
	\draw[fermion] (-1,2) -- (0,2);
	\draw[fermion] (0,2) -- (1,2.5);
	\draw[vector] (0,2) -- (.5,1);
	% \draw[fermion] (0,0) -- (.5,1);
	\draw[fermion] (.5,1) -- (1.5,1);
	\draw[fermion] (60:.3) -- (.5,1);
	%
	\draw[fermion] (-1,.2) -- (0,.2);
	\draw[fermion] (-1,.1) -- (0,.1);
	\draw[fermion] (-1,0) -- (0,0);
	%
	\draw[fermion] (0,-.1) -- (1.5,-0.1);
	\draw[fermion] (0,-.2) -- (1.5,-0.2);
	\fill[white] (0,0) circle (.3);
	\begin{scope}
    	\clip (0,0) circle (.3cm);
    	\foreach \x in {-.9,-.8,...,.3}
			\draw[line width=1 pt] (\x,-.3) -- (\x+.6,.3);
  	\end{scope}
	\draw[fermionnoarrow] (0,0) circle (.3);
 \end{tikzpicture}

\vspace{1cm}

\begin{tikzpicture}[line width=1.5 pt, scale=2]
	\draw[fermion] (-1,2.1) -- (-.2,2.1);
	\draw[fermion] (-1,2) -- (-.2,2);
	\draw[fermion] (-1,1.9) -- (-.2,1.9);
	%
	\draw[fermion] (0,2) -- (2,2);
	\draw[fermion] (0,2.1) -- (2,2.1);
	\draw[fermion] (0,2) -- (.5,1);
	% \draw[fermion] (0,0) -- (.5,1);
	\draw[vector] (.5,1) -- (1,1);
	\draw[fermion] (2,1.5) -- (1,1);
	\draw[fermion] (1,1) -- (2,.5);
	%
	\draw[fermion] (60:.3) -- (.5,1);
	%
	\draw[fermion] (-1,.1) -- (-.2,.1);
	\draw[fermion] (-1,0) -- (-.2,0);
	\draw[fermion] (-1,-.1) -- (-.2,-.1);
	%
	\draw[fermion] (0,0) -- (2,0);
	\draw[fermion] (0,-.1) -- (2,-.1);
	\fill[white] (0,0) circle (.3);
%
	\begin{scope}
    	\clip (0,0) circle (.3cm);
    	\foreach \x in {-.9,-.8,...,.3}
			\draw[line width=1 pt] (\x,-.3) -- (\x+.6,.3);
  	\end{scope}
	\draw[fermionnoarrow] (0,0) circle (.3);
%	
	\begin{scope}[shift={(0,2)}]
		\fill[white] (0,0) circle (.3);
    	\clip (0,0) circle (.3cm);
    	\foreach \x in {-.9,-.8,...,.3}
			\draw[line width=1 pt] (\x,-.3) -- (\x+.6,.3);
		\draw[fermionnoarrow] (0,0) circle (.3);
  	\end{scope}
	\draw[fermionnoarrow] (0,2) circle (.3);
 \end{tikzpicture}


\vspace{1em}


\begin{center}
\begin{tikzpicture}[line width=1.5 pt]
	\node at (-2.5,0) {$K^+$};
	\node at (1.8,1.8) {$\pi^+$};
	\node at (2.4,1.8) {$\pi^-$};
	\node at (4.7,1.2) {$e$};
	\node at (4.7,-1.2) {$\nu$};
	\node at (3,-.6) {$A_\mu$};
	\node at (1.5,-.3) {$K^+$};
	\draw (-2,0) -- (0,0) -- (3,0);
	\draw[fermion] (4.5,1) -- (3,0);
	\draw[fermion] (3,0) -- (4.5,-1);
	\draw[scalarnoarrow] (-.2,0) -- (1.8,1.5);
	\draw[scalarnoarrow] (.2,0) -- (2.2,1.5);
	\draw[fill=black] (0,0) circle (.3cm);
	\draw[fill=white] (0,0) circle (.29cm);
	\begin{scope}
    	\clip (0,0) circle (.3cm);
    	\foreach \x in {-1.0,-.8,...,.4}
			\draw[line width=1 pt] (\x,-.3) -- (\x+.6,.3);
  	\end{scope}
	\begin{scope}[shift={(3,0)}]
		\draw[fill=black] (0,0) circle (.3cm);
		\draw[fill=white] (0,0) circle (.29cm);
		\begin{scope}
	    	\clip (0,0) circle (.3cm);
	    	\foreach \x in {-1.0,-.8,...,.4}
				\draw[line width=1 pt] (\x,-.3) -- (\x+.6,.3);
	  	\end{scope}
	\end{scope}
 \end{tikzpicture}
\end{center}



\section{SUSY Cascade}



\begin{tikzpicture}[line width=2 pt, scale=2]
	\draw (-1,1) -- (-.22cm,.22cm);
	\draw (-1,-1) -- (-.22cm,-.22cm);
	%
	\draw[line width=.75 pt] (.22cm,.22cm) -- (1,1);
	\draw[line width=.75 pt] (.22cm,-.22cm) -- (1,-1);
	\draw[gluon, line width=1 pt] (.22cm,.22cm) -- (1,1);
	\draw[gluon, line width=1 pt] (.22cm,-.22cm) -- (1,-1);
	\node at (65:1) {$\tilde{g}$};
	\node at (-65:1) {$\tilde{g}$};
	%
	\begin{scope}
    	\clip (0,0) circle (.3cm);
    	\foreach \x in {-.9,-.8,...,.3}
			\draw[line width=1 pt] (\x,-.3) -- (\x+.6,.3);
  	\end{scope}
  	\draw[fermionnoarrow] (0,0) circle (.3);
  	%
  	%\draw[scalarnoarrow] (1,1) -- (2,1);
  	%\draw[scalarnoarrow] (1,-1) -- (2,-1);
  	%
  	\begin{scope}[shift={(1,1)}]
	  	\node at (63:.6) {$t/\bar t$};
	  	\node at (-50:.6) {$\tilde t$};
  		\draw[fermionnoarrow] (0,0) -- (45:1);
  		\draw[scalarnoarrow] (0,0) -- (-30:1);
  		\begin{scope}[shift={(45:1)}]
  			\draw[fermionnoarrow] (0,0) -- (45:1);
  			\node at (45:1.2) {$b$};
  			\draw[vector] (0,0) -- (1,0);
  			\node at (.7,.2) {$W$};
  			\begin{scope}[shift={(1,0)}]
  				\draw[fermionnoarrow] (0,0) -- (15:1);
  				\draw[fermionnoarrow] (0,0) -- (-15:1);
  				\node at (15:1.2) {$\ell$};
  				\node at (-15:1.2) {$\nu$};
  			\end{scope}
  		\end{scope}
  		\begin{scope}[shift={(-30:1)}]
  			\draw[fermionnoarrow] (0,0) -- (30:1);
  			\node at (30:1.2) {$b$};
  			\draw[fermionnoarrow] (0,0) -- (-0:1);
  			\node at (0:1.2) {$s$};
  		\end{scope}
  	\end{scope}
  	\begin{scope}[shift={(1,-1)}]
  		\node at (-68:.6) {$t/\bar t$};
  		\node at (50:.6) {$\tilde t$};
   		\draw[fermionnoarrow] (0,0) -- (-45:1);
  		\draw[scalarnoarrow] (0,0) -- (30:1);
  		\begin{scope}[shift={(-45:1)}]
  			\draw[fermionnoarrow] (0,0) -- (-45:1);
  			\node at (-45:1.2) {$b$};
  			\draw[vector] (0,0) -- (1,0);
  			\node at (.7,-.2) {$W$};
  			\begin{scope}[shift={(1,0)}]
  				\draw[fermionnoarrow] (0,0) -- (15:1);
  				\draw[fermionnoarrow] (0,0) -- (-15:1);
  				\node at (-15:1.2) {$\ell$};
  				\node at (15:1.2) {$\nu$};
  			\end{scope}
  		\end{scope}
  		\begin{scope}[shift={(30:1)}]
  			\draw[fermionnoarrow] (0,0) -- (0:1);
  			\node at (0:1.2) {$b$};
  			\draw[fermionnoarrow] (0,0) -- (-30:1);
  			\node at (-30:1.2) {$s$};
  		\end{scope}
  	\end{scope}
\end{tikzpicture}




\vspace{2em}


\begin{tikzpicture}[line width=2 pt, scale=2]
	\draw (-1,1) -- (-.22cm,.22cm);
	\draw (-1,-1) -- (-.22cm,-.22cm);
	%
	\draw[scalar] (.22cm,.22cm) -- (1,1);
	\draw[scalarbar] (.22cm,-.22cm) -- (1,-1);
	\node at (65:1) {$\tilde{t}_2$};
	\node at (-65:1) {$\tilde{t}_2$};
	%
	\begin{scope}
    	\clip (0,0) circle (.3cm);
    	\foreach \x in {-.9,-.8,...,.3}
			\draw[line width=1 pt] (\x,-.3) -- (\x+.6,.3);
  	\end{scope}
  	\draw[fermionnoarrow] (0,0) circle (.3);
  	%
  	%\draw[scalarnoarrow] (1,1) -- (2,1);
  	%\draw[scalarnoarrow] (1,-1) -- (2,-1);
  	%
  	\begin{scope}[shift={(1,1)}]
	  	\node at (63:.6) {$\tilde t_1$};
	  	\node at (-50:.6) {$Z$};
  		\draw[scalar] (0,0) -- (45:1);
  		\draw[vector] (0,0) -- (-30:1);
  		\begin{scope}[shift={(45:1)}]
  			\draw[fermion] (0,0) -- (1,0);
  			\node at (45:1.2) {$\tilde G$};
  			\draw[line width=.75 pt] (0,0) -- (45:1);
  			\draw[vector, line width=1 pt] (0,0) -- (45:1);
  			\node at (.6,.2) {$t$};
  			\begin{scope}[shift={(1,0)}]
  				\draw[vector] (0,0) -- (30:1);
  				\draw[fermion] (0,0) -- (-15:1);
  				\node at (-15:1.2) {$b$};
  				\node at (50:0.5) {$W$};
  				\begin{scope}[shift={(30:1)}]
  					\draw[fermion] (0,0) -- (15:.8);
	  				\draw[fermionbar] (0,0) -- (-15:.8);
	  				\node at (-15:1) {$\ell$};
	  				\node at (15:1) {$\nu$};
  				\end{scope}
  			\end{scope}
  		\end{scope}
  		\begin{scope}[shift={(-30:1)}]
  			\draw[fermion] (0,0) -- (15:1);
  			\node at (15:1.2) {$\ell$};
  			\draw[fermionbar] (0,0) -- (-5:1);
  			\node at (-5:1.2) {$\ell$};
  		\end{scope}
  	\end{scope}
  	%
  	% Lower half
  	%
  	\begin{scope}[shift={(1,-1)}]
  		\node at (-68:.6) {$\tilde t_1$};
  		\node at (50:.6) {$Z$};
   		\draw[scalarbar] (0,0) -- (-45:1);
  		\draw[vector] (0,0) -- (30:1);
  		\begin{scope}[shift={(-45:1)}]
  			% \draw[fermionnoarrow] (0,0) -- (-45:1);
  			\draw[line width=.75 pt] (0,0) -- (-45:1);
  			\draw[vector, line width=1 pt] (0,0) -- (-45:1);
  			\node at (-45:1.2) {$\tilde G$};
  			\draw[fermionbar] (0,0) -- (1,0);
  			\node at (.6,-.2) {$t$};
  			\begin{scope}[shift={(1,0)}]
  				\draw[vector] (0,0) -- (-30:1);
  				\draw[fermion] (0,0) -- (15:1);
  				\node at (15:1.2) {$b$};
  				\node at (-50:0.5) {$W$};
  				\begin{scope}[shift={(-30:1)}]
  					\draw[fermion] (0,0) -- (15:.8);
	  				\draw[fermionbar] (0,0) -- (-15:.8);
	  				\node at (-15:1) {$\nu$};
	  				\node at (15:1) {$\ell$};
  				\end{scope}
  			\end{scope}
  		\end{scope}
  		\begin{scope}[shift={(30:1)}]
  			\draw[fermion] (0,0) -- (15:1);
  			\node at (-5:1.2) {$\ell$};
  			\draw[fermionbar] (0,0) -- (-5:1);
  			\node at (15:1.2) {$\ell$};
  		\end{scope}
  	\end{scope}
\end{tikzpicture}





\section{Multilepton stops}


\begin{tikzpicture}[line width=1.5 pt, scale=1.6]
	\draw (-.75,.75) -- (-.22cm,.22cm);
	\draw (-.75,-.75) -- (-.22cm,-.22cm);
	%
	\draw[scalar] (.22cm,.22cm) -- (1,.22);
	\draw[scalarbar] (.22cm,-.22cm) -- (1,-.22);
	\draw[fill] (1.2,-.3) circle (.03);
	\draw[fill] (1.4,-.3) circle (.03);
	\draw[fill] (1.6,-.3) circle (.03);
	\node at (.6,.5) {$\tilde{t}_2$};
	\node at (.6,-.5) {$\tilde{t}_2$};
	%
	\begin{scope}
    	\clip (0,0) circle (.3cm);
    	\foreach \x in {-.9,-.8,...,.3}
			\draw[line width=1 pt] (\x,-.3) -- (\x+.6,.3);
  	\end{scope}
  	\draw[fermionnoarrow] (0,0) circle (.3);
  	%
%  	%
  	\begin{scope}[shift={(1,.22)}]
	  	\node at (70:.6) {$Z$};
	  	\node at (.5,-.25) {$\tilde t_1$};
  		\draw[scalar] (0,0) -- (0:1);
  		\draw[vector] (0,0) -- (45:1);
  		\begin{scope}[shift={(1,0)}]
  			\draw[fermion] (0,0) -- (1,0);
  			\node at (45:1.2) {$\tilde G$};
  			\draw[line width=.75 pt] (0,0) -- (45:1);
  			\draw[vector, line width=1 pt] (0,0) -- (45:1);
  			\node at (.5,-.25) {$t$};
  			\begin{scope}[shift={(1,0)}]
  				\draw[vector] (0,0) -- (1,0);
  				\draw[fermion] (0,0) -- (45:1);
  				\node at (45:1.2) {$b$};
  				\node at (.5,-0.25) {$W$};
  				\begin{scope}[shift={(1,0)}]
  					\draw[fermion] (0,0) -- (15:.8);
	  				\draw[fermionbar] (0,0) -- (-15:.8);
	  				\node at (-15:1) {$\ell$};
	  				\node at (15:1) {$\nu$};
  				\end{scope}
  			\end{scope}
  		\end{scope}
  		\begin{scope}[shift={(45:1)}]
  			\draw[fermion] (0,0) -- (55:.75);
  			\node at (55:.9) {$\ell$};
  			\draw[fermionbar] (0,0) -- (35:.75);
  			\node at (35:.9) {$\ell$};
  		\end{scope}
  	\end{scope}
%  	%
%  	% Lower half
%  	%
%  	\begin{scope}[shift={(1,-1)}]
%  		\node at (-68:.6) {$\tilde t_1$};
%  		\node at (50:.6) {$Z$};
%   		\draw[scalarbar] (0,0) -- (-45:1);
%  		\draw[vector] (0,0) -- (30:1);
%  		\begin{scope}[shift={(-45:1)}]
%  			% \draw[fermionnoarrow] (0,0) -- (-45:1);
%  			\draw[line width=.75 pt] (0,0) -- (-45:1);
%  			\draw[vector, line width=1 pt] (0,0) -- (-45:1);
%  			\node at (-45:1.2) {$\tilde G$};
%  			\draw[fermionbar] (0,0) -- (1,0);
%  			\node at (.6,-.2) {$t$};
%  			\begin{scope}[shift={(1,0)}]
%  				\draw[vector] (0,0) -- (-30:1);
%  				\draw[fermion] (0,0) -- (15:1);
%  				\node at (15:1.2) {$b$};
%  				\node at (-50:0.5) {$W$};
%  				\begin{scope}[shift={(-30:1)}]
%  					\draw[fermion] (0,0) -- (15:.8);
%	  				\draw[fermionbar] (0,0) -- (-15:.8);
%	  				\node at (-15:1) {$\nu$};
%	  				\node at (15:1) {$\ell$};
%  				\end{scope}
%  			\end{scope}
%  		\end{scope}
%  		\begin{scope}[shift={(30:1)}]
%  			\draw[fermion] (0,0) -- (15:1);
%  			\node at (-5:1.2) {$\ell$};
%  			\draw[fermionbar] (0,0) -- (-5:1);
%  			\node at (15:1.2) {$\ell$};
%  		\end{scope}
%  	\end{scope}
\end{tikzpicture}



\section{Strongly coupled blob}
	Uses \texttt{tikzlibrary\{calc\}}.
	\vspace{1em}
	
	
	\begin{tikzpicture}[line width=1.5,scale=1]
	\pgfmathsetmacro{\offset}{2.5}
	
	\begin{scope}
	    \clip[rounded corners=1cm] (60:1) 
    -- ($(60:1)+(0:\offset)$)
    -- ($(60:1)+(0:\offset)-(0,2*sin(60)$) 
    -- (-60:1)
    -- ($(-60:1)+(240:\offset)$)
    -- ($(180:1)+(240:\offset)$)
    -- (180:1)
    -- ($(180:1)+(120:\offset)$)
    -- ($(60:1)+(120:\offset)$)
    -- cycle;
	   \foreach \x in {-5,-4.8,...,3}
		\draw[line width=.8 pt] (\x,-3) -- (\x+2,3);
	\end{scope}
	
    \draw[
    line width=3, 
    draw=black,
    rounded corners=1cm]
    (60:1) 
    -- ($(60:1)+(0:\offset)$)
    -- ($(60:1)+(0:\offset)-(0,2*sin(60)$) 
    -- (-60:1)
    -- ($(-60:1)+(240:\offset)$)
    -- ($(180:1)+(240:\offset)$)
    -- (180:1)
    -- ($(180:1)+(120:\offset)$)
    -- ($(60:1)+(120:\offset)$)
    -- cycle;
%    (1,0) -- ($(1,0)-(0,.5)$) -- (0,0) -- cycle;
%       (0,0) -- (0,1) -- (1,1) -- cycle;
\end{tikzpicture}
	
	
	\url{http://tex.stackexchange.com/questions/48756/tikz-relative-coordinates}
	
	
	\vspace{1em}
    \begin{tikzpicture}[line width=1.5,scale=.6]
	\pgfmathsetmacro{\offset}{2.5}
	\pgfmathsetmacro{\rounding}{15}
	
	\begin{scope}
	    \clip[rounded corners=\rounding] (60:1) 
    -- ($(60:1)+(0:\offset)$)
    -- ($(60:1)+(0:\offset)-(0,2*sin(60)$) 
    -- (-60:1)
    -- ($(-60:1)+(240:\offset)$)
    -- ($(180:1)+(240:\offset)$)
    -- (180:1)
    -- ($(180:1)+(120:\offset)$)
    -- ($(60:1)+(120:\offset)$)
    -- cycle;
	   \foreach \x in {-5,-4.8,...,3}
		\draw[line width=.8 pt] (\x,-3) -- (\x+2,3);
	\end{scope}
	
    \draw[
    line width=2, 
    draw=black,
    rounded corners=\rounding]
    (60:1) 
    -- ($(60:1)+(0:\offset)$)
    -- ($(60:1)+(0:\offset)-(0,2*sin(60)$) 
    -- (-60:1)
    -- ($(-60:1)+(240:\offset)$)
    -- ($(180:1)+(240:\offset)$)
    -- (180:1)
    -- ($(180:1)+(120:\offset)$)
    -- ($(60:1)+(120:\offset)$)
    -- cycle;

% http://tex.stackexchange.com/questions/130838/tikz-segments-with-5pt-dotted-ends
    \draw[line width=2.5, line cap=round, loosely dashed] ($(0:1)+(0:0.9*\offset)$) -- ($(0:1)+(0:2*\offset)$);
    
    \draw[fermionbar, line width=2.5, line cap=round] ($(120:1)+(120:0.8*\offset)$) -- ($(120:1)+(120:1.5*\offset)$);
    
    \draw[fermion, line width=2.5, line cap=round] ($(-120:1)+(-120:0.8*\offset)$) -- ($(-120:1)+(-120:1.5*\offset)$);

\end{tikzpicture}

\vspace{1em}
	
\begin{tikzpicture}[line width=1.5,scale=.6]
	\pgfmathsetmacro{\offset}{2.5}
	\pgfmathsetmacro{\rounding}{15}
	
	
	\draw[fermion, line width=2.5, line cap=round] (-6,0) -- (-2.5,0);
		
	 \draw[
    line width=2, 
    fill=white,
    rounded corners=\rounding]
    (120:1) 
    -- ($(120:1)-(0:\offset)$)
    -- ($(120:1)-(0:\offset)-(0,2*sin(60)$) 
    [sharp corners]
    -- (-120:1)
%    -- ($(-120:1)+(120:1)$)
    -- cycle;
	
	\begin{scope}
	    \clip[rounded corners=\rounding] (120:1) 
    -- ($(120:1)-(0:\offset)$)
    -- ($(120:1)-(0:\offset)-(0,2*sin(60)$) 
    [sharp corners]
    -- (-120:1)
    -- cycle;
	   \foreach \x in {-5,-4.8,...,3}
		\draw[line width=.8 pt] (\x,-3) -- (\x+2,3);
	\end{scope}
	
    \draw[
    line width=2, 
    draw=black,
    rounded corners=\rounding]
    (120:1) 
    -- ($(120:1)-(0:\offset)$)
    -- ($(120:1)-(0:\offset)-(0,2*sin(60)$) 
    [sharp corners]
    -- (-120:1)
%    -- ($(-120:1)+(120:1)$)
    -- cycle;

% http://tex.stackexchange.com/questions/130838/tikz-segments-with-5pt-dotted-ends


\end{tikzpicture}


\vspace{1em}
	
\begin{tikzpicture}[line width=1.5,scale=.6]
	\pgfmathsetmacro{\offset}{2.5}
	\pgfmathsetmacro{\rounding}{15}
	


	\draw[line width=2.5, line cap=round, loosely dashed] (-6,0) -- (-2.5,0);
		
	 \draw[
    line width=2, 
    fill=white,
    rounded corners=\rounding]
    (120:1) 
    -- ($(120:1)-(0:\offset)$)
    -- ($(120:1)-(0:\offset)-(0,2*sin(60)$) 
    [sharp corners]
    -- (-120:1)
%    -- ($(-120:1)+(120:1)$)
    -- cycle;
	
	\begin{scope}
	    \clip[rounded corners=\rounding] (120:1) 
    -- ($(120:1)-(0:\offset)$)
    -- ($(120:1)-(0:\offset)-(0,2*sin(60)$) 
    [sharp corners]
    -- (-120:1)
    -- cycle;
	   \foreach \x in {-5,-4.8,...,3}
		\draw[line width=.8 pt] (\x,-3) -- (\x+2,3);
	\end{scope}
	
    \draw[
    line width=2, 
    draw=black,
    rounded corners=\rounding]
    (120:1) 
    -- ($(120:1)-(0:\offset)$)
    -- ($(120:1)-(0:\offset)-(0,2*sin(60)$) 
    [sharp corners]
    -- (-120:1)
%    -- ($(-120:1)+(120:1)$)
    -- cycle;

% http://tex.stackexchange.com/questions/130838/tikz-segments-with-5pt-dotted-ends


\end{tikzpicture}



\section{Cat Diagram}

From Georgi--Kaplan on vacuum misalignment.
\vspace{1em}

	\begin{tikzpicture}[line width=1.5 pt, scale=2.5]
		
		\draw[fermionbar, line width=1.2 pt] (57:.6) -- (85:.29);
		\draw[fermion, line width=1.2 pt] (57:.6) -- (29:.29);

		
		\draw[fermionbar, line width=1.2 pt] (123:.6) -- (151:.29);
		\draw[fermion, line width=1.2 pt] (123:.6) -- (95:.29);
		
		\draw[vector, line width=1.2pt] (123:.6) arc (123:57:.6);
		
		\draw[fermionnoarrow, line width=1.2pt] (0,0) -- (170:.7);
		\draw[fermionnoarrow, line width=1.2pt] (0,0) -- (180:.7);
		\draw[fermionnoarrow, line width=1.2pt] (0,0) -- (190:.7);
		
		\draw[fermionnoarrow, line width=1.2pt] (0,0) -- (10:.7);
		\draw[fermionnoarrow, line width=1.2pt] (0,0) -- (0:.7);
		\draw[fermionnoarrow, line width=1.2pt] (0,0) -- (-10:.7);


       \begin{scope}		
		\foreach \x in {205,215,...,335}
		    \draw[fill=black] (\x:.7) circle (.01);
       \end{scope}
      
       % 		\draw[fill=white] (0,0) circle (.29cm);
		
		\draw[fill=black] (0,0) circle (.3cm);
		\draw[fill=white] (0,0) circle (.295cm);
		\begin{scope}
	    	\clip (0,0) circle (.3cm);
	    	\foreach \x in {-.95,-.9,-.85,...,.3}
				\draw[line width=.8 pt] (\x,-.3) -- (\x+.6,.3);
	  	\end{scope}
	 \end{tikzpicture}
	 
	 



\section*{Acknowledgements}


This work is supported in part by the \textsc{nsf} grant \textsc{phy}-1316792. 
%
%\textsc{p.t.}\ thanks 
%\emph{your name here}
%for useful comments and discussions. 

%% Appendices
% \appendix

% \bibliographystyle{utphys} 
% \bibliography{bib title without .bib}

\end{document}